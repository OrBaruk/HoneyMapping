% Institute of Computing - University of Campinas
% Example using the ic-tese-v2 class
% See the contributors for this class in the file CONTRIBUTORS.txt
% Before you starts using this class read the file README.txt
% Doubts and problems mail info-cpg at ic dot unicamp dot br

\documentclass[%
        % Use this line to thesis with more than 100 pages in ARABIC NUMERATION.
        % Or remove it to thesis with up to 100 pages in ARABIC NUMBERATION.
        TwoSidePages,%
        % Select one of the following idioms to be the main language of your thesis.
        % If you remove this line the main language will be English.
        English,% Spanish % Portuguese
        % This line set your thesis up as the reviewed version after the defense.
        % Remove this line if you have not defended yet.
        FinalVersion,%
        % If you intend to use a copyright page, check with CPG or your advisor
        % before. There is complex legal questions with strong consequences.
        Copyright,%
        % Let this command to show the list of your tables. Or removed it.
        TablesPage,%
        % Let this command to show the list of your figures. Or removed it.
        FiguresPage,%
        % You have to use coadvisor's command to set the second supervisor's name.
        % Use this command only if you have done a work with co-supervision
        % (co-tutela) at foreign universities.
%       CoTutela]%
        ]%
{ic-tese-v2}

\usepackage[latin1, utf8]{inputenc}
% If you want to include PDF identification remove the comments in the following lines.
\usepackage
%               [pdfauthor={nome do autor},
%                pdftitle={titulo},
%                pdfkeywords={palavras, chaves},
%                pdfproducer={Latex with hyperref},
%                pdfcreator={pdflatex}]
{hyperref}

\begin{document}
%##############################################################################%
%Your work definitions
        \thetitle%
        {The title of your work in English: This is a very long example with subtitle. O puede ser un largo t\'itulo en espa\~nol.}
        {Escreva aqui o t\'itulo em portugu\^es da sua disserta\c{c}\~ao/tese. Note que ser\'a este t\'itulo que aparecer\'a quando selecionar o idioma Portugu\^es.}

        \thisyear{2014}
        \author{John D. Smith}
        \degreesoughtpt{Doutorado} % Mestrado
        \degreesought{DEGREE} % PhD/MSc/Doctorado/Mag\'ister

        \titlesoughtpt{Doutor} % Mestre
        \titlesought{TITLE} % Doutor/Doutora/Mestre/Maestro/Maestra/PhD/Doctor(somente para o espanhol)/Doctora

        %\dept{Department/\texit{Portuguese Department}} Your departament. This command has not been used for while.

        \principaladvisor{Prof. Dr. Fulano Orientador Silva}
        \advisortitle{Supervisor/\textit{Orientador}} % Orientadora/Director/Directora
        % If you does have co-advisor comment the command.
        \coadvisor{Prof. Dr. Sicrano Braga} %Co-orientadora  , pode ser omitido

        % This command has effect only when CoTutela is used.
        % If you are using this command to specify your second supervisor, you have to write
        % in this fashion: \coadvisortitle{(Supervisor/\textit{Orientador})\\Massachusetts Institute of Technology - MIT}
        \coadvisortitle{(Supervisor/\textit{Orientador})} % Orientadora/Director/Directora

        \firstreader{Prof. Dr. Member 2\\Institute of Computing - UNICAMP}
        \secondreader{Prof. Dr. Member 3\\Institute of Computing - UNICAMP}
        \thirdreader{Dr. External Member 4\\Institute X}
        \fourthreader{Dr. External Member 5\\Institute X or Y}
        \fifthreader{\ldots}
        \sixthreader{Prof. Dr. Member 6\\Institute of Computing - UNICAMP (Substitute/\textit{Suplente})}
        \seventhreader{Prof. Dr. Member 7\\Institute of Computing - UNICAMP (Substitute/\textit{Suplente})}

        \grants{{\rm Financial support: CNPq scholarship (process XYZ) 2010--2012}} % Pode ser omitido.

        \defencedate{15}{06}{2013} % Use the format \defencedate{day}{month}{year}.

        % If you are using the copyright page, set the year of it.
        \copyrightyear{2015}

        % To include the PDF with the library register (ficha catalográfica),
        % use \fichacatalografia{arquivo}. The file extension can be omitted.
        %\fichacatalografica{doc.pdf}

        % To include the PDF with the examiner board signatures (assinatura da banca),
        % use \assinaturabanca{arquivo}. The file extension can be omitted.
        %\assinaturabanca{doc.pdf}

        % To include the PDF with the advisor's signature (assinatura do orientador na folha de rosto),
        % use \folhaderosto{arquivo}. The file extension can be omitted.
        %\folhaderosto{}

        % This command have to be used right here.
        \beforepreface

        \begin{theabstract}
        The abstract must contain at most 500 words...\\

\textbf{Observation:} the text in this page *has to* fit it in. The Unicamp's printing service would reject this work
if more than one page were used.
        \end{theabstract}

        % Enable this only when Spanish is the main language.
%%%%%%%%%%%%%%%%%%%%%%%%%%%%%%%%%%%%%%%%%%%%%%%%%%%%%%%%%%%%%%%%%%%%%%%%%%%%%%%%
%       \begin{elresumen}
%        El resumen debe contener un m{\'a}ximo de 500 palabras...\\
%
%\textbf{Observation:} the text in this page *has to* fit it in. The Unicamp's printing service would reject this work
%if more than one page were used.
%       \end{elresumen}
%%%%%%%%%%%%%%%%%%%%%%%%%%%%%%%%%%%%%%%%%%%%%%%%%%%%%%%%%%%%%%%%%%%%%%%%%%%%%%%%
        \begin{oresumo}
        O resumo deve conter no m{\'a}ximo 500 palavras...\\

\textbf{Observation:} the text in this page *has to* fit it in. The Unicamp's printing service would reject this work
if more than one page were used.
        \end{oresumo}

        % Example of dedication.
%%%%%%%%%%%%%%%%%%%%%%%%%%%%%%%%%%%%%%%%%%%%%%%%%%%%%%%%%%%%%%%%%%%%%%%%%%%%%%%%
%       \begin{dedico}{Dedication}
%       \textit{
%          To somebody that is very special to me.
%        }
%       \end{dedico}
%%%%%%%%%%%%%%%%%%%%%%%%%%%%%%%%%%%%%%%%%%%%%%%%%%%%%%%%%%%%%%%%%%%%%%%%%%%%%%%%

        % Keep this section in a single page to avoid thesis rejection from PRPG.
    \prefacesection{Acknowledgements} % Acknowledgements/Agradecimientos/Agradecimentos
 I would like to thank\ldots
\textbf{Aten\c{c}\~ao:} os agradecimentos *t\^em* de caber em uma p\'agina impressa. Eventual estouro
 poder\'a causar rejei\c{c}\~ao pela Gr\'afica, j\'a que a pagina\c{c}\~ao sai fora do padr\~ao.

        % Example of epigraph.
%%%%%%%%%%%%%%%%%%%%%%%%%%%%%%%%%%%%%%%%%%%%%%%%%%%%%%%%%%%%%%%%%%%%%%%%%%%%%%%%
%       \begin{dedico}{Epigraph}
%        \textit{
%            ``If you read this work, you will be able to read the work of everyone else.''
%        }\\
%        \raggedleft A philosophical person
%       \end{dedico}
%%%%%%%%%%%%%%%%%%%%%%%%%%%%%%%%%%%%%%%%%%%%%%%%%%%%%%%%%%%%%%%%%%%%%%%%%%%%%%%%


        % If you have some kind of list that has to be after the list of figures, you should active it into this command.
        % You do not use this command more than once. Active all your list at once into the command.
%%%%%%%%%%%%%%%%%%%%%%%%%%%%%%%%%%%%%%%%%%%%%%%%%%%%%%%%%%%%%%%%%%%%%%%%%%%%%%%%
        %\mylist{
% Include here your lists of acronyms, glossary, etc. They will be shown after the list of figures.
% Note that an alphabetic index is usually placed after the references.
        %}
%%%%%%%%%%%%%%%%%%%%%%%%%%%%%%%%%%%%%%%%%%%%%%%%%%%%%%%%%%%%%%%%%%%%%%%%%%%%%%%%

        % As the beforepreface do not misplace this command.
    \afterpreface % Gera: Conteudo, Lista de Tabelas, Lista de Figuras.

%##############################################################################%
%                                                                              %
%                                                                              %
%                                                                              %
%                       Your thesis starts right here.                         %
%                                                                              %
%                                                                              %
%                                                                              %
%##############################################################################%
\chapter{Introduction}
Computer science became\ldots

They represent a sliver of the electorate yet could still hold the key to the Oval Office.

In a contest that's already the most expensive in history, we set out to meet the men and women whose choices are so highly prized: the undecided voters.
They represent six key groups in the key swing states where their votes matter most:
A millennial in New Hampshire. A Catholic in Ohio. A long-term unemployed man in Nevada. A Latino in Florida. A single woman in Virginia. An evangelical in Iowa.

We introduced you to them through deeply drawn profiles accompanied by photographic portraits, videos and data visualizations that illuminate the nexus between real life and politics -- the emotional terrain that determines how these ordinary Americans could decide the election.

With less than a day left before the election, we caught up with them to see if they've made up their minds -- and how they came to their decision. Three told whom they're voting for, two wouldn't say and one -- well, she's still undecided.

The last few months have been especially busy for Tyler York. He's worked 60 hours a week between his three jobs, and he welcomed two nephews into his already big, bustling family.
But the 25-year-old spent a lot of time thinking about the election, too, trying to decide how to decide. He watched the debates and talked to supporters of President Barack Obama and Mitt Romney, but he wasn't satisfied with anybody's answers.

Tyler York: ``I don't need to be reassured of my own opinion. ... Self-reflection is the most important part (in deciding).''
``People my age, people generations older ... I've realized that they don't truly know why they're voting for their guy,'' he said. '' 'You should vote for me because the other guy's bad' isn't really educating anybody on the issues.''
A colleague suggested this: Imagine you are the candidate. What are the issues most important to you? What is your stance? And which of the real-life candidates lines up best? Not perfectly, but best?

Read York's story: 'What's going on? Am I really happy?'
He spent hours researching the candidates' ideas on foreign policy, a subject he's passionate about after traveling overseas and making friends among the large refugee community around Manchester, New Hampshire.

He read up on energy and the environment, which sparked his interest after teaching a sustainability class at a private school. He was stunned by how little information he found about education; there's a lot of talk about its importance, but no vision from candidates, he thinks. And he studied their stances on women's health, gender pay disparities and same-sex marriage.

York, the undecided millennial, became York, the decided voter -- but he doesn't want to say who he'll support. Sifting through so many out-of-context attacks and meme-making one-liners made him wish for clearer facts, not more noisy opinions.
He still identifies as an independent and can't imagine himself volunteering for a campaign or cause in the future. The moment you do, he said, you stop listening to other sides' points of view, or everyone thinks you do. Obama and Romney are often pretty similar, he thinks. York's not so sure the divisions in politics and parties are real. It's just that loud arguments and snappy comebacks get attention.

That doesn't mean it's easy for any candidate to bring people together, he said.
``If it was a legitimate difference or true barrier, it would be one thing,'' York said. ``But because it's manufactured, it feels like 'I'm not supposed to agree with you.' ''
York said he'll spend the days before the election talking to trusted friends and family members whose votes are going the opposite way. He's not sure they could change his mind, but he wants to listen.

``I want to know why they feel so strongly in that direction,'' he said. ``I hope to hear an eloquent argument of why they feel that way.
``I don't need to be reassured of my own opinion. It takes a lot more than watching news or reading articles (to make the decision.) Self-reflection is the most important part.''
He's making decisions for himself, too. By this time next year, he still expects to be working all three of his part-time jobs. He's enjoying them, so why stop? He doesn't think he'll still be living above his parents' garage, though. He plans to have a stable income for more than a year by then, and he'll have saved for a healthy down payment on a place of his own -- if he decides to buy.

But he's still not sure he wants to make the commitment.
The Catholic: Looking for the truth
Mary Roberts usually waits until she's attended rallies for both political parties to make up her mind about how she'll vote. Living in the battleground state of Ohio, she has a greater chance of doing that than most Americans -- except for this year.

Mary Roberts: ``I always like when people start a sentence with, 'I'm going to tell you the truth.' I really start to listen then.''
That's because someone recently hit her car, and it was totaled. Now she attends physical therapy and needs a cane, which has left the normally active 67-year-old feeling frustrated. It's also slowed her decision-making process.
``I tried to go to (Ohio State University) one day since both candidates were having rallies there,'' Roberts says. ``But I couldn't get close enough to comfortably walk. I'm so disappointed.''
Which means, just days from this year's election, she still hasn't made up her mind.
``I'm close, real close,'' she says.
Instead, she must rely on the information that comes to her.
Read Roberts' story: She acts in faith. Who deserves hers?
``Let me count,'' Roberts says, as she flips through her stack of mail. ``I've got seven different political mailers just today.''
She doesn't mind the mail, nor does she mind the calls -- except when there are so many they fill up her voice mail and her family can't leave a message.
``And I'd love to watch a game show without the 50 political ads,'' she says. ``My sister complains she feels neglected living in Georgia. But the ads are so negative, I tell her she's lucky.'' (You should take a look in the Table \ref{Tab:example}). This has nothing about that, but it is a small example of citation \cite{smallest_example2013}.

\begin{table}
        \center
        \caption{Table to show an example of Table.}
        \label{Tab:example}
        \begin{tabular}{c|c}
                \hline
                        Column 1 & Column 2\\
                \hline
                        My data 1 & My data 2\\
                \hline
        \end{tabular}
\end{table}

\section{What this is all about}

What it is this? It's it!

\section{What a problem!}

\subsection{Hard one}

\subsection{Hard two}

\subsection{Hard three}

\subsection{Hard four}

\subsection{Hard five}

\section{What a solution!}

\subsection{Nice one}

\subsection{Nice two}

\subsection{Nice three}

\subsection{Nice four}

\subsection{Nice five}

\chapter{Conclusions} %Conclusions/Conclusiones/Conclus\~oes
This work\ldots

\section{This must be the way}

\section{Surely is!}

\subsection{Hard one}

\subsection{Hard two}

\subsection{Hard three}

\subsection{Hard four}

\subsection{Hard five}

\section{Cheers!}

\subsection{Nice one}

\subsection{Nice two}

\subsection{Nice three}

\subsection{Nice four}

\subsection{Nice five}

%Bibliography always comes before the appendix. Do not move it.
\bibliographystyle{plain}
\bibliography{mybib}
\appendix
\chapter{Demonstration}
\ldots
%Annex only here.
\end{document}
